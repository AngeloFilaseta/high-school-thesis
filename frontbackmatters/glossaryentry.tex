\newglossaryentry{Emulare}{
name=Emulare,
text={emulare},
description={Rendere possibile il funzionamento di un certo sistema su uno diverso dall'originale}
}

\newglossaryentry{opensource}{
name={Open Source },
text={open source},
description={Termine usato per specifici software non protetti da copyright e liberamente modificabile da altri utenti della rete oltre al creatore}
}

\newglossaryentry{claw}{
name=Claw Crane,
text={claw crane},
description={La macchina presente in sale giochi, supermercati e cinema che permettono la vincita di premi grazie all'uso di una gru ad artiglio}
}

\newglossaryentry{achievement}{
name=Achievement,
text={achievement},
description={Un riconoscimento attribuito per aver compiuto una determinata impresa in un videogame}
}

\newglossaryentry{f2p}{
name=Free2Play,
text={Free2Play},
description={Un videogioco che permette il suo intero contenuto (o quasi) gratuitamente}
}

\newglossaryentry{crt}{
name= CRT,
description={Schermo a tubo catodico.}
}

\newglossaryentry{mmorpg}{
name=MMORPG,
description={\textit{Massive Multiplayer Online Role-Playing Game}, un gioco di ruolo multigiocatore online giocato contemporaneamente da più persone reali che condividono lo stesso mondo virtuale}
}

\newglossaryentry{esport}{
name=eSport,
description={Gli sport elettronici che vengono giocati a livelli competitivi e organizzati, proprio come delle olimpiadi}
}

\newglossaryentry{fps}{
name=FPS,
description={ Da non confondere con la sigla \textit{Frame per second}. \textit{First Person Shooter} è una categoria di videogiochi sparatutto in prima persona}
}

\newglossaryentry{e3}{
name=E3,
description={Abbreviazione di \textit{Electronic Entertainment Expo}, è la fiera di videogiochi più famosa del mondo}
}

\newglossaryentry{jamma}{
name=JAMMA,
description={\textit{Japan Amusement Machinery Manufacturers Association}, un'associazione che si occupa di raccogliere i vari produttori di videogiochi arcade. È però anche il vecchio standard utilizzato dalle vecchie macchina da sala giochi per collegare tutti gli apparecchi hardware}
}

\newglossaryentry{snk}{
name=SNK,
description={\textit{Shin Nihon Kikaku}, azienda giapponese produttrice di videogiochi 2D arcade e 3D per console}
}

\newglossaryentry{c}{
name=C++,
description={Linguaggio di programmazione orientato agli oggetti migliorato rispetto al predecessore, il linguaggio C}
}