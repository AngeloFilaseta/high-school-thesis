
\begin{thesisabstract}[english]
Col tempo, tutto è destinato a subire un cambiamento, un’evoluzione. L’uomo moderno ha un irrefrenabile bisogno di progredire, ma non sempre il risultato finale può eguagliare in bellezza quello sperato. A volte possono capitare momenti bui e di crisi; la nostra storia ne è la dimostrazione pratica. Quando succede qualcosa del genere, c’è sempre qualcuno che pagherebbe oro per tornare indietro. Lo stesso è accaduto per i videogiochi. Con l'avvento di internet, il mondo intero ha cominciato a considerare i videogame come un ottimo passatempo su cui molte aziende, soprattutto al giorno d’oggi, hanno un business. Proprio queste ultime hanno cercato di accontentare più persone possibili offrendo loro l’esperienza desiderata. È un vero peccato che la maggior parte dei cambiamenti che ci sono stati nell’ultimo decennio abbiano portato ad un peggioramento rispetto al passato. L’esperienza videoludica di un tempo era un incredibile mix di avventure e storie mozzafiato. Andando avanti nel tempo i videogiochi hanno cambiato forma, acquisendo molti aspetti caratteristici della cinematografia. Questi ultimi però non sono neanche lontanamente paragonabili ai videogiochi che hanno segnato l’infanzia di milioni di persone. È proprio per questo che ultimamente sta avvenendo una specie di ritorno alle origini. Molti sviluppatori, ai giorni d’oggi, forniscono appositamente una grafica scadente ai loro giochi rispetto agli standard per ricordare loro le esperienze passate.\\Lo Scopo di questa tesina è tornare indietro nel tempo, grazie alla costruzione di un cabinato arcade casalingo completamente gratuito. La macchina finale permette di giocare a qualsiasi gioco (anche se piuttosto datato) di qualsiasi piattaforma e con i giusti controlli proprio come in una vecchia sala giochi. C’è una differenza sostanziale rispetto al passato però. I tempi sono cambiati, alcune tecnologie si sono evolute o sono diverse da come erano un tempo. Per la realizzazione del progetto sono stati usati metodi moderni e, anche esteticamente, è impossibile non notare il contrasto tra passato e presente.
\end{thesisabstract}

\begin{thesisabstract}[english]
As time goes by, everything changes and evolves. The modern person is always looking for a way to progress, but at times, the expected final result doesn’t fulfill the goal. A concrete demonstration of this concept may be the dark times of our history. When something like this happens, there’s always someone that would be willing to go back in time.  This was the case in the videogame world: with the advent of the internet, the entire world started considering videogames as an excellent hobby. A lot of companies, especially nowadays, have made a business of them. These companies are  doing their best to satisfy the highest number of people as possible, by giving them the experience they’re searching for. It is really sad that the changes occurred in this last decade have led to a worsening in comparison to the past. The old videogame experience was an incredible mix of breathtaking adventures. In their evolution, videogames acquired many aspects from the cinematography world. These games are completely different from the kind of videogames that millions of people treasure. That’s why videogames are returning to their original state. Many developers purposely implement poor graphics in their games to evoke the feel of past experiences; a return to their gaming roots.\\This thesis’ objective is to build a free homemade arcade cabinet. The modernized machine allows us to play any game (even if old) from any platform and with the proper controls. There’s a great difference from the past because some technologies have changed with time. In this project, contemporary methods have been used to achieve a result that highlights the contrast between the past and present.
\end{thesisabstract}
