\chapter{Conclusione}
\section{Difficoltà Riscontrate}
\begin{quotation}\small
“Uno partecipa a un esperimento sapendo di poter fallire, ma uno non prende parte a un esperimento sapendo di aver fallito.”\\-R. \& R. Lutece (Bioshock Infinite)
\end{quotation}
 Il progetto è stato molto lungo e faticoso da portare a termine. Non riscontrare difficoltà lungo l’arco del lavoro sarebbe stato improbabile se non impossibile. I cambiamenti da fare nel mobile erano parecchi e alcuni sono stati abbastanza problematici. Lo scomparto del monitor non era progettato per supportare schermi piatti, ma solo schermi CRT. Il legno assumeva infatti una forma curva nella parte interna rendendo l’inserimento di un monitor LCD impossibile. Inoltre, mantenendo quel pezzo curvo, sarebbe stato possibile vedere i bordi dello schermo dietro al legno. Il falegname di fiducia non ha avuto problemi ad “appiattire” il pezzo e inizialmente sembravano non esserci più problemi. Quando però il monitor è stato smontato, il pannello è risultato più grande del previsto di qualche millimetro in tutti i lati. È stato necessario portare nuovamente il pezzo al falegname per eseguire un’ulteriore lavorazione, ma poi l’incastro è riuscito.
Un altro problema strutturale ha riguardato la lavorazione della plancia. Sopra il pezzo di legno della plancia in cui sono stati praticati i fori dal falegname è stato disposto un pezzo con le stesse dimensioni ma in plexiglass, avente spessore di 3mm. Inizialmente però, i buchi sul legno e i buchi sul plexiglass non combaciavano. Per fare in modo che succedesse è stato necessario allargare i fori del legno con una lima, ma dato che ad ogni azione corrisponde una reazione, questo ha causato uno sviluppo indesiderato. Nelle immediate vicinanze del primo tasto del giocatore uno si può notare una piccola scavatura nel legno proprio sotto il plexiglass. Per fortuna, grazie allo spacer del bottone, il difetto è quasi invisibile a causa del contrasto causato dalla luminosità del led sulla vernice e sulla plastica nera.\\Quattro bottoni sono stati posizionati troppo in alto quando è stata progettata la plancia, causando un'inaspettata divergenza contro il metallo. È stato necessario tagliare alcuni pezzi del metallo stesso per risolvere il problema.\\Durante la fase di assemblaggio, la striscia di led si è spezzata in due passando attraverso un buco troppo stretto. Per fortuna esistono dei punti in cui si può praticare un’incisione ed effettuare una saldatura dei quattro contatti e quindi ripararla al prezzo di qualche centimetro di essa.\\La peggiore problematica riscontrata però è stata data dal vecchio PC usato. In realtà inizialmente il computer da usare era completamente diverso da quello attuale. Possedeva un Pentium 4 da 2,2GHz, ben 2GB di memoria RAM e una scheda video AMD. L’installazione del sistema operativo era riuscita, l’installazione dell’emulatore anche. I giochi funzionavano perfettamente e non ci sarebbero dovuti essere problemi nell’esportazione del computer nel cabinato. In pratica era tutto pronto e finito. All’ultima accensione di prova però è comparso il peggiore degli imprevisti. Sul monitor non arrivava il segnale, e non solo quello del cabinato; nessun tipo di monitor era in grado di stampare a video qualcosa. Si è poi scoperto che la colpa era della motherboard, nella quale erano scoppiati dei condensatori. È stato necessario sostituire tutti i componenti con altri comunque adatti ad adempire le funzioni che da svolgere.
\section{Sviluppi Futuri}
\begin{quotation}\small
“Un uomo deve fare della sua vita una scala che non smette mai di scalare -- Se non stai salendo, stai scivolando giù per i gradini, amico mio.”\\-John Shirley (BioShock: Rapture)
\end{quotation}
 Nonostante funzioni tutto perfettamente è possibile effettuare tante piccole migliorie o cambiamenti che renderebbero il lavoro molto più apprezzabile. Anzitutto, è possibile reinserire in qualsiasi momento una gettoniera se si desidera far funzionare il cabinato come in una normale sala giochi. Se mai quindi il cabinet volesse essere usato per fruttare guadagno tramite il sistema a gettoni, sarebbe possibile farlo cambiando i collegamenti sulla scheda di encoding. È infatti possibile collegare la gettoniera al pin “COIN” al posto dei bottoni attuali. Una gettoniera d’altronde non è altro che un apparecchio che manda segnali elettrici (proprio come un microswitch) al passaggio del gettone. In questo caso si otterrebbero anche ulteriori due bottoni che possono acquisire funzioni speciali come la pausa o l’uscita dal gioco.\\È possibile in qualsiasi momento espandere e ripartizionare l’Hard Disk. Al momento è presente solo un sistema operativo (Linux) ma potrebbe essere utile aggiungerne altri. Alcuni software o giochi per esempio funzionano solo su un sistema Windows. Si potrebbe eseguire un upgrade al sistema audio del cabinato per farlo funzionare anche come Jukebox. Ci sarebbe una soluzione molto pratica anche se non si volesse inserire la musica nel disco rigido o dovesse occupare troppo spazio. Tramite una chiavetta USB Bluetooth è possibile accoppiare qualunque dispositivo Android o iOS e riprodurre la musica sulle casse direttamente dal dispositivo. Se preferibile è possibile anche usare un semplicissimo cavo AUX.\\
Riguardo l’audio, è possibile fare una modifica un pò più complessa per quanto riguarda l’alimentazione dei led. Collegando i due impianti, sarebbe possibile aumentare l’intensità della luminosità dei diodi in base al volume per ottenere un effetto molto particolare.\\Esistono alcuni giochi che richiedono una periferica diversa rispetto a Joystick e pulsanti. Le light gun sono veri e propri controller a forma di arma da fuoco utilizzati in diversi giochi sparatutto (come il celebrissimo\textit{ Duck Hunt}). Le vecchie pistole ottiche sono dotate di un fotodiodo nella canna, il quale riesce a rilevare le differenze di luce sullo schermo e quindi capire se il bersaglio è stato abbattuto o no. Questo principio però non funziona con monitor LCD; per questo sono state create light gun che funzionano grazie a sensori infrarossi posti vicino allo schermo. Si potrebbero aggiungere questi sensori e procedere all’acquisto di una pistola per rendere possibile anche giocare a questo tipo di videogame.
